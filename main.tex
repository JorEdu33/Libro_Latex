% ------------------------------------------------------------------------
%
% -------------------      Plantilla_UIS.tex       -----------------------
%
% ------------------------------------------------------------------------
% ------------------------------------------------------------------------
% ------------------------------------------------------------------------
% Versión de plantilla para realización de informes de trabajo de grado
% construida para uso de la Universidad Industrial de Santander.
%
% Reservados todos los derechos
%
% Bucaramanga, Colombia
%
% Febrero 03 de 2019
%
% ------------------------------------------------------------------------
% ------------------------------------------------------------------------
% ------------------------------------------------------------------------
%
% --------------------------\documentclass[letter,oneside,12pt,spanish]{report}

\documentclass[letter,oneside,12pt,spanish]{report}
% ------------------------------------------------------------------------
\usepackage{uislatexstyleAPA} % Libreria UIS APA
% ------------------------------------------------------------------------

% --- AQUÍ VAN TODOS LOS PAQUETES --- %
\usepackage[utf8]{inputenc}
\usepackage{epsfig}
\usepackage{amsmath}
\usepackage{amssymb}
\usepackage{graphicx}
\usepackage{float}
\usepackage{multirow}
\usepackage{wrapfig}
\usepackage{enumitem}
\usepackage{xcolor}
\usepackage{caption}
\usepackage{tcolorbox}
\usepackage{subfigure}
\usepackage{listings}
\usepackage{minted}
\usepackage{csquotes}
\usepackage[style=apa]{biblatex}
\usepackage{geometry}
\usepackage{hyperref}

% --- CONFIGURACIONES DE COLORES --- %
\definecolor{bg}{rgb}{0.95,0.95,0.92}
\definecolor{purple}{rgb}{0.5,0,0.5}
\definecolor{orange}{rgb}{0.9,0.4,0}
\definecolor{codeborder}{RGB}{0,0,0}
\definecolor{codebackground}{RGB}{255,255,255}
\definecolor{codegray}{RGB}{128,128,128}
\definecolor{codegreen}{RGB}{0,153,0}
\definecolor{codeorange}{RGB}{255,125,80}

% --- CONFIGURACIÓN DE LISTINGS --- %
\lstdefinestyle{mystyle}{
	frame=shadowbox,
	backgroundcolor=\color{codebackground},
	commentstyle=\color{codegray},
	keywordstyle=\color{codeorange},
	numberstyle=\tiny\color{codegray},
	stringstyle=\color{codegreen},
	basicstyle=\ttfamily\footnotesize,
	breakatwhitespace=false,
	breaklines=true,
	captionpos=b,
	keepspaces=true,
	numbers=none,
	showspaces=false,
	showstringspaces=false,
	showtabs=false,
	tabsize=2,
	frame=single,
	rulecolor=\color{codeborder},
	framesep=4pt,
	xleftmargin=10pt,
	xrightmargin=10pt,
}
\lstset{style=mystyle}

% --- OTRAS CONFIGURACIONES --- %
\addbibresource{biblio.bib}
\geometry{headheight=15.13202pt}
\captionsetup[figure]{labelfont={bf}, singlelinecheck=false}
\parskip=12pt

\begin{document}                              % Inicio de documento
% ------------------------------------------------------------------------
% Definición silábica de palabras
% ------------------------------------------------------------------------
% \hyphenation{pro-por-cio-nal di-se-ño}
\hypersetup {pdfborder = {0 0 0}}
% ------------------------------------------------------------------------
% Titulo resumido del trabajo que aparecerá en cornisa
% ------------------------------------------------------------------------
\title{AMBIENTE DE APRENDIZAJE PARA LA ENSEÑANZA DE ESTADÍSTICA II}
% ------------------------------------------------------------------------
% Elementos previos al contenido del trabajo
% ------------------------------------------------------------------------ 

\begin{center}

Ambiente de aprendizaje para la enseñanza de Estadística II: un enfoque en teoría, prácticas y autoevaluación\vspace{1.5cm}

Jorge Eduardo Suárez Cortés\\
Daniel Alejandro Sánchez Rodríguez \\ \vspace{1.5cm}

Trabajo de Grado para optar al título de Ingeniero de Sistemas\\ \vspace{1.5cm}

Director\\
Andrés Leonardo González Gómez\\
PhD (c). Ciencias de la Computación\vspace{1.5cm}


Universidad Industrial de Santander\\
Facultad de Ingenierías Fisicomecánicas\\
Escuela de Ingeniería de Sistemas e Informática\\
Ingeniería de Sistemas\\
Bucaramanga\\
2025\\

\end{center}

% Portadilla

\parskip=0pt

\newpage

\titlecontents{paragraph}[0em]{\normalfont\normalsize}{\thecontentslabel. }{}{\titlerule*[1em]{}\contentspage}



\parskip=12pt
\tableofcontents                    

% Tabla de contenido
% ------------------------------------------------------------------------
\newpage

\listoffigures                         % Lista de figuras, tablas y anexos
\newpage
\listoftables
%\listofanexo
% ------------------------------------------------------------------------
\newpage

\chapter*{Glosario}

\begin{description}

\item \textbf{Ambiente Virtual de Aprendizaje (AVA)}: Entorno digital que integra recursos, actividades y herramientas interactivas para apoyar los procesos de enseñanza y aprendizaje en línea o semipresenciales.
\item \textbf{Autoevaluación}: Estrategia pedagógica que permite a los estudiantes valorar su propio proceso de aprendizaje mediante actividades de retroalimentación automática.
\item \textbf{Base de Datos}: Conjunto estructurado de información almacenada de manera organizada que permite el acceso, la gestión y la actualización de datos académicos y de evaluación.
\item \textbf{Docker}: Plataforma de virtualización ligera que permite empaquetar aplicaciones y servicios en contenedores portables, facilitando su despliegue en diferentes entornos.
\item \textbf{Estadística Inferencial}: Rama de la estadística que utiliza datos muestrales para realizar inferencias, estimaciones o pruebas sobre parámetros poblacionales.
\item \textbf{Google Colab}: Herramienta en la nube que permite ejecutar código en Python y R desde un navegador, facilitando la realización de prácticas y análisis de datos sin necesidad de instalación local.
\item \textbf{Grader}: Sistema automatizado de evaluación que corrige ejercicios prácticos, especialmente de programación y estadística, generando resultados inmediatos.
\item \textbf{LMS (Learning Management System)}: Sistema de gestión del aprendizaje que permite planificar, implementar y evaluar procesos educativos en entornos virtuales. Facilita la creación de cursos, administración de contenidos, seguimiento del progreso de los estudiantes y la comunicación entre docentes y alumnos.
\item \textbf{Moodle}: Plataforma de gestión de aprendizaje (LMS, por sus siglas en inglés) de código abierto, utilizada para crear cursos en línea, administrar contenidos y facilitar la comunicación entre docentes y estudiantes.
\item \textbf{Node.js}: Entorno de ejecución de JavaScript orientado al desarrollo de aplicaciones web y servidores, utilizado para gestionar la comunicación entre sistemas del entorno virtual.
\item \textbf{Playit.gg}: Servicio que permite crear túneles seguros para acceder a aplicaciones locales desde internet, sin necesidad de configuraciones avanzadas en redes.

\end{description}


% ------------------------------------------------------------------------
\newpage

% Contenido del Informe
% ------------------------------------------------------------------------
\newpage

\chapter*{Resumen}

\footnotesize{
\begin{description}
  \item[Título:] Ambiente de aprendizaje para la enseñanza de Estadística II: un enfoque en teoría, prácticas y autoevaluación.
  \astfootnote{Trabajo de grado}
  \item[Autores:] Jorge Eduardo Suárez Cortés y Daniel Alejandro Sánchez Rodríguez.
  \asttfootnote{Facultad de Ingenierías Fisicomecánicas. Escuela de Ingeniería de Sistemas e Informática.\\
  	Director: Andrés Leonardo Gonzáles Gómez}
  \item[Palabras Clave:]Ambiente virtual de aprendizaje, Moodle, Estadística II, Autoevaluación.
  \item[Descripción:] El presente trabajo de grado describe el diseño, desarrollo e implementación 
  de un ambiente de aprendizaje virtual para la asignatura Estadística II, dirigida a estudiantes de 
  la Escuela de Ingeniería de Sistemas e Informática de la Universidad Industrial de Santander. El 
  ambiente fue construido sobre la plataforma Moodle, con el objetivo de ofrecer una experiencia de 
  aprendizaje más activa, significativa y continua mediante la integración sistemática de contenidos 
  teóricos, ejercicios prácticos y mecanismos de evaluación formativa, tales como autoevaluaciones y 
  retroalimentación inmediata.

  El enfoque pedagógico del entorno se basa en el aprendizaje activo y la construcción progresiva del 
  conocimiento, promoviendo la participación continua del estudiante y el desarrollo de competencias 
  relacionadas con el análisis estadístico y la interpretación de datos. Las actividades propuestas están 
  diseñadas para favorecer la autonomía del aprendizaje. Los resultados obtenidos sugieren que el uso del
  ambiente virtual contribuye a mejorar la comprensión de los contenidos estadísticos, así como a incrementar 
  la motivación y el compromiso de los estudiantes con su proceso de formación. Esta propuesta representa 
  una alternativa metodológica y tecnológica que responde a las necesidades actuales de la educación superior
  en el ámbito de la ingeniería de sistemas, especialmente en contextos donde la articulación entre teoría y 
  práctica resulta esencial para el desarrollo de competencias profesionales.
  
  
\end{description}}\normalsize

% ------------------------------------------------------------------------

\newpage

\chapter*{Abstract}

\footnotesize{
\begin{description}
  \item[Title:] Learning Environment for the Teaching of Statistics II: A Focus on Theory, Practice, and Self-Assessment.
  \astfootnote{Degree Work}
  \item[Authors:] Jorge Eduardo Suárez Cortés y Daniel Alejandro Sánchez Rodríguez.
  \asttfootnote{Faculty of Physicomechanical Engineering. School of Systems and Computer Engineering.\\
  	Director: Andrés Leonardo Gonzáles Gómez}
  \item[Key words:] Virtual learning environment, Moodle, Statistics II, Self-assessment.
  \item[Description:]This thesis describes the design, development, and implementation 
  of a virtual learning environment for the Statistics II course, aimed at students of 
  the School of Systems Engineering and Computer Science at the Industrial University of Santander. The 
  environment was built on the Moodle platform, with the aim of offering a more active, meaningful, and 
  continuous learning experience through the systematic integration of theoretical content, practical 
  exercises, and formative assessment mechanisms, such as self-assessments and immediate feedback.

  The pedagogical approach of the environment is based on active learning and the progressive construction of 
  knowledge, promoting continuous student participation and the development of skills 
  related to statistical analysis and data interpretation. The proposed activities are 
  designed to encourage autonomous learning. The results obtained suggest that the use of the
  virtual environment contributes to improving the understanding of statistical content, as well as increasing 
  student motivation and commitment to their training process. This proposal represents 
  a methodological and technological alternative that responds to the current needs of higher education
  in the field of systems engineering, especially in contexts where the articulation between theory and 
  practice is essential for the development of professional skills.

\end{description}}\normalsize

% Abstract

% ------------------------------------------------------------------------
% Capítulos
% ------------------------------------------------------------------------
\newpage

\nnchapter{Introducción}



% ------------------------------------------------------------------------

\newpage


\chapter{Objetivos}

\section{Objetivo General}

\begin{itemize}
    \item Diseñar y desarrollar un entorno virtual para la asignatura de Estadística II, de la Escuela de Ingeniería de Sistemas e 
    Informática de la UIS, que facilite la ejecución de un plan de aula propuesto para la práctica, el aprendizaje y la autoevaluación 
    del contenido de la asignatura, mediante herramientas de Python y R.
\end{itemize}

\section{Objetivos Especificos}

\begin{itemize}
    \item Diseñar un plan de aula modular para la asignatura Estadística II que organice los contenidos teóricos y prácticos en 
    unidades interactivas basadas en Colab Notebooks, Python y R, e incluya calificadores automáticos y ejercicios con retroalimentación 
    instantánea.
    
    \item Implementar un ambiente de aprendizaje virtual en la plataforma Moodle que integre las tecnologías definidas (Colab Notebooks, 
    Python y R) y organice los contenidos teóricos y prácticos junto con materiales didácticos de apoyo, consolidando una experiencia 
    formativa práctica y accesible para la asignatura Estadística II.
    
    \item Establecer instrumentos de medición para las actividades prácticas y autoevaluativas en Moodle (desarrolladas con Python y R), 
    que permitan medir el logro de los resultados de aprendizaje durante el desarrollo del proyecto en el curso de Estadística II.

    \item Pilotear el ambiente virtual de aprendizaje diseñado, con sus contenidos interactivos e instrumentos de evaluación, en por lo 
    menos dos grupos de Estadística II de la Escuela de Ingeniería de Sistemas e Informática de la UIS, con el fin de validar su 
    usabilidad, satisfacción y efectividad en el logro de los resultados de aprendizaje.

\end{itemize}

% ------------------------------------------------------------------------

\newpage

\chapter{Marco Teorico}

\section{Estadística}

La estadística constituye una herramienta fundamental en la ciencia, la ingeniería y la administración, al proporcionar métodos rigurosos para recopilar, organizar, presentar, analizar e interpretar datos con el fin de apoyar la toma de decisiones. En la actualidad, desempeña un papel crucial en el análisis de sistemas complejos, la mejora de procesos y el control de calidad en distintos entornos académicos, industriales y científicos.

De acuerdo con \textcite{montgomery1996}, la estadística moderna se divide en dos grandes áreas: estadística descriptiva y estadística inferencial. La primera se ocupa de técnicas que permiten resumir y describir de manera gráfica o numérica los datos disponibles, mientras que la segunda se orienta hacia la formulación de generalizaciones o la toma de decisiones sobre una población a partir de la información contenida en una muestra.

En el contexto de este proyecto, la atención se centra en la estadística inferencial, dado que la asignatura Estadística II aborda conceptos como la estimación puntual, la construcción de intervalos de confianza, las pruebas de hipótesis y el estudio de distribuciones muestrales. Dichos contenidos resultan esenciales para obtener conclusiones válidas a partir de datos incompletos o parciales. Al tratarse de una asignatura que combina teoría matemática con aplicaciones prácticas en problemas reales, se requiere no solo comprender los fundamentos conceptuales, sino también desarrollar competencias aplicadas. Por ello, este proyecto plantea el diseño de un entorno virtual interactivo que permita a los estudiantes ejercitar dichos conceptos con datos reales, actividades prácticas y retroalimentación inmediata.

\subsection{Estadística Inferencial}

La estadística inferencial es considerada una de las ramas más potentes y aplicadas de la disciplina. Su propósito central es extraer conclusiones sobre una población a partir de la información proporcionada por una muestra. Según \textcite{montgomery1996}, la mayoría de las aplicaciones estadísticas en ciencia, ingeniería y administración incorporan procedimientos de inferencia y toma de decisiones.

Entre los conceptos más relevantes que sustentan esta disciplina se encuentran:

\begin{itemize}
	\item \textbf{Distribuciones muestrales}: distribuciones de probabilidad de estadísticas como la media, la proporción o la varianza, que permiten describir la variabilidad esperada de los estimadores a partir de distintas muestras.
	
	\item \textbf{Estimación de parámetros}: procedimientos mediante los cuales se obtienen valores aproximados de parámetros poblacionales utilizando la información muestral.
	
	\item \textbf{Pruebas de hipótesis}: métodos formales que permiten aceptar o rechazar proposiciones acerca de parámetros poblacionales, basándose en la evidencia empírica extraída de una muestra.
\end{itemize}

\section{Ambientes de aprendizaje}

\subsection{Políticas TIC en educación}

El diseño de ambientes de aprendizaje en la educación superior no solo responde a fundamentos pedagógicos y tecnológicos, sino también a marcos normativos e institucionales que orientan su desarrollo. En este sentido, la Política Institucional de Tecnologías de la Información y la Comunicación (TIC) de la Universidad Industrial de Santander, establecida mediante el Acuerdo del Consejo Superior No. 051 de 2009, constituye un referente clave para la integración de recursos digitales en los procesos académicos.

Dicha política busca garantizar el acceso equitativo a la información, fomentar la innovación pedagógica y promover la incorporación de tecnologías abiertas e interoperables en la formación universitaria. Asimismo, establece como principios fundamentales el fortalecimiento de los procesos de enseñanza-aprendizaje, la inclusión de toda la comunidad académica en el uso de las TIC y la consolidación de una cultura institucional que aproveche estas herramientas para el mejoramiento continuo de la calidad educativa \parencite{uis2009}.

De esta forma, el presente proyecto se enmarca en los lineamientos de la Política TIC, ya que aprovecha plataformas como Moodle y Google Colab para potenciar la enseñanza de la estadística mediante recursos digitales interactivos, autoevaluaciones y retroalimentación automática. Esto permite articular los objetivos pedagógicos con los principios institucionales de innovación, inclusión y acceso abierto al conocimiento.

\subsection{Ambiente de aprendizaje}

El concepto de ambiente de aprendizaje se refiere al conjunto de condiciones, espacios, interacciones y recursos que permiten y favorecen el desarrollo de procesos educativos efectivos. Este término alude a un escenario dinámico en el que los individuos desarrollan capacidades, competencias, habilidades y valores. Es decir, no se trata de un espacio fijo ni neutral, sino de un entorno que debe transformarse en función de las necesidades de los estudiantes y de las innovaciones educativas \parencite{castro2019}.

En este sentido, un ambiente de aprendizaje no se limita al aula física, sino que constituye una construcción pedagógica en la que intervienen aspectos sociales, culturales, tecnológicos y metodológicos. De acuerdo con \textcite{castro2019}, un ambiente de aprendizaje debe ser:

\begin{itemize}
	\item Flexible y adaptable a diferentes contextos y tecnologías.
	\item Fomentar la interacción social y la construcción colectiva del conocimiento.
	\item Proporcionar recursos adecuados, incluidos los tecnológicos, que potencien las competencias estudiantiles.
	\item Promover el rol activo del docente como diseñador, facilitador y mediador del aprendizaje.
\end{itemize}

Desde esta perspectiva, los ambientes de aprendizaje se conciben como espacios dinámicos e integrales, que van más allá de la transmisión de información y buscan estimular la participación activa, la autonomía y la construcción colaborativa del conocimiento.

\subsection{Ambientes virtuales de aprendizaje}

El avance de las tecnologías digitales ha posibilitado la transición de los entornos tradicionales hacia los Ambientes Virtuales de Aprendizaje (AVA). Estos se definen como plataformas tecnológicas que integran recursos didácticos, actividades de aprendizaje, espacios de interacción y mecanismos de evaluación para facilitar procesos formativos en entornos digitales \parencite{salinas2004}.

En coherencia con las características planteadas por \textcite{castro2019}, los AVA son flexibles y adaptables a diferentes contextos, fomentan la interacción social y colaborativa, proporcionan recursos tecnológicos que amplían las competencias estudiantiles y posicionan al docente como un mediador y facilitador del aprendizaje.

Uno de los sistemas más representativos en este campo es Moodle, un Learning Management System (LMS) de código abierto que ha sido adoptado globalmente en educación superior debido a su enfoque pedagógico constructivista, su escalabilidad y su facilidad de integración con otras herramientas \parencite{dougiamas2003}. Moodle permite gestionar cursos, estudiantes, actividades y evaluaciones, convirtiéndose en un eje articulador del proceso formativo.

En el contexto del presente proyecto, Moodle constituye el espacio institucional de referencia en la UIS, donde los estudiantes acceden a materiales de teoría, actividades interactivas y prácticas en Google Colab, además de las autoevaluaciones conectadas con el backend. Su importancia radica en que integra de manera armónica la teoría, la práctica y la retroalimentación automática, consolidando un entorno virtual de aprendizaje efectivo.



\section{Tecnologías implementadas en el proyecto}

\subsection{Node.js y Express.js}

Node.js es un entorno de ejecución basado en el motor V8 de Google Chrome que permite ejecutar JavaScript en el lado del servidor, diseñado bajo un modelo de I/O no bloqueante. Esto lo convierte en una herramienta adecuada para aplicaciones que requieren manejar múltiples solicitudes concurrentes, como plataformas de aprendizaje en línea \parencite{tilkov2010}.

Express.js es un framework minimalista para Node.js que facilita la construcción de APIs y aplicaciones web mediante la organización de rutas, middleware y controladores \parencite{brown2019}. En el proyecto, Node.js y Express.js cumplen el rol de backend principal, gestionando la validación de respuestas de estudiantes, el control de intentos y la comunicación con la base de datos MariaDB.

\subsection{React.js}

React.js es una biblioteca de JavaScript desarrollada por Facebook que permite construir interfaces de usuario basadas en componentes reutilizables \parencite{banks2017}. Su arquitectura con virtual DOM optimiza la actualización de vistas, mejorando la eficiencia y la experiencia de usuario.

El frontend, desarrollado en React.js, centraliza la gestión de calificaciones y el control de intentos de los estudiantes. Su principal ventaja es la transparencia: ante cualquier discrepancia con la calificación calculada por el backend (Node.js), un instructor puede revisar tanto el código enviado por el estudiante como la nota obtenida. Esto facilita una verificación rápida y fundamentada, garantizando la equidad en la evaluación.

\subsection{Bases de datos relacionales (MariaDB)}

MariaDB es un sistema de gestión de bases de datos relacional de código abierto, derivado de MySQL, que organiza los datos en tablas relacionadas bajo el modelo relacional propuesto por \textcite{codd1970}.

Su función en el proyecto es ser el repositorio central de información, donde se almacenan los registros de estudiantes, intentos de ejercicios, calificaciones y respuestas. Gracias a sus características de integridad y consistencia, MariaDB asegura la fiabilidad en el almacenamiento de los resultados generados en los procesos de autoevaluación.

\subsection{Virtualización y contenedores (Docker)}

Docker es una plataforma de contenedores ligeros que empaqueta aplicaciones y dependencias en un entorno aislado, lo que garantiza portabilidad y reproducibilidad \parencite{merkel2014}.

En el proyecto, Docker permite desplegar servicios como el backend, frontend, playit.gg y la base de datos en un entorno controlado, asegurando que las pruebas piloto y los despliegues sean consistentes sin importar el sistema operativo. Su uso facilita la escalabilidad y reduce los problemas de compatibilidad entre entornos de desarrollo y producción.

\subsection{Moodle como LMS}

Moodle es un sistema de gestión de aprendizaje (LMS) de código abierto ampliamente utilizado en la educación superior. Ofrece herramientas para gestionar cursos, usuarios, recursos y actividades de evaluación \parencite{dougiamas2003}.

En este proyecto, Moodle funciona como el entorno de acceso institucional provisto por la UIS, donde los estudiantes encuentran los enlaces a los notebooks de Google Colab, las actividades prácticas y los recursos teóricos. Moodle sirve como el punto de integración pedagógica que articula los recursos tecnológicos desarrollados.

\subsection{Python y R}

Python y R son lenguajes de programación consolidados en el campo del análisis estadístico y la ciencia de datos. Python, gracias a bibliotecas como NumPy, SciPy y Pandas, se ha convertido en una herramienta versátil para la enseñanza y aplicación de métodos estadísticos \parencite{mckinney2017}. R, por su parte, es un lenguaje especializado en estadística y visualización de datos, ampliamente usado en contextos académicos \parencite{rcoreteam2023}.

En el proyecto, para el desarrollo de los graders se emplearán ambos lenguajes de programación, lo que permitirá dotarlos de una mayor versatilidad en su uso.

\subsection{Google Colab}

Google Colab es una plataforma en la nube que permite ejecutar notebooks de Python sin necesidad de instalar software localmente. Combina teoría, código ejecutable y resultados en un único entorno interactivo \parencite{bisong2019}.

Su integración en el proyecto permite a los estudiantes resolver ejercicios en tiempo real, ejecutar simulaciones y enviar sus respuestas al backend para validación. Además, Colab facilita el aprendizaje activo y autónomo.

\subsection{Graders y autoevaluación automatizada}

Los graders son sistemas de evaluación automática que comparan respuestas de estudiantes con soluciones esperadas, otorgando calificaciones y retroalimentación inmediata \parencite{kurnia2001}. Herramientas como nbgrader o Otter-Grader han demostrado su efectividad en la enseñanza de programación y estadística.

En este proyecto, los graders están implementados en el backend desarrollado con Node.js y conectados a los notebooks de Colab. De esta forma, los estudiantes reciben retroalimentación instantánea, lo que fomenta el aprendizaje autónomo y reduce la carga de corrección manual para los docentes.

\subsection{Playit.gg y tunelización de servicios}

Playit.gg es una herramienta que permite crear túneles seguros entre una máquina local y la web pública, facilitando la exposición de servicios locales sin necesidad de configuración compleja de red.

Durante el desarrollo y pruebas piloto del proyecto, Playit.gg se utiliza para exponer el backend y el frontend alojados en contenedores Docker hacia los estudiantes y notebooks de Colab. Esta estrategia permite realizar pilotos de manera controlada.


% ------------------------------------------------------------------------
\newpage


\chapter{Estado del Arte}

\section{Introducción}
La enseñanza de la estadística en la educación superior enfrenta desafíos significativos debido a la abstracción de sus conceptos y a la necesidad de aplicar la teoría a problemas del mundo real. En este contexto, los Ambientes Virtuales de Aprendizaje (AVA) y las plataformas digitales se han posicionado como recursos clave para mediar en los procesos formativos, permitiendo integrar materiales, actividades prácticas y evaluaciones. Sin embargo, aunque las soluciones existentes han avanzado en accesibilidad y organización de contenidos, aún se evidencian limitaciones relacionadas con la retroalimentación inmediata, la integración de entornos de análisis de datos y la evaluación automatizada de procesos complejos como los de la estadística inferencial.

\section{Antecedentes}

\subsection{Plataformas de Aprendizaje y Moodle}
Las plataformas de gestión del aprendizaje (LMS) como Moodle continúan siendo un estándar en la educación digital. Recientes estudios destacan que los estudiantes valoran la accesibilidad y la claridad organizativa de Moodle, mientras que los docentes reconocen su carácter intuitivo y adaptable a distintos escenarios educativos \parencite{pacheco2025}. Sin embargo, se identifican limitaciones en contextos de baja infraestructura tecnológica, donde la dependencia de conectividad estable puede afectar su adopción \parencite{ndibalema2025}.

Adicionalmente, las analíticas de aprendizaje basadas en el uso del LMS se han convertido en un campo emergente. Por ejemplo, la persistencia y constancia de los estudiantes en el uso de la plataforma se correlaciona directamente con el éxito académico, lo que abre la puerta a modelos predictivos de intervención temprana \parencite{goh2025}. Estos resultados evidencian que Moodle no solo cumple funciones administrativas, sino que puede ser una fuente de información valiosa para mejorar la enseñanza de la estadística cuando se complementa con herramientas externas de análisis y práctica computacional.

\subsection{Sistemas de Autoevaluación y Retroalimentación Automatizada}
La retroalimentación inmediata es uno de los factores determinantes en la consolidación del aprendizaje de estadística. Mientras que los entornos tradicionales de evaluación suelen centrarse en respuestas cerradas, las herramientas modernas de autoevaluación automatizada han logrado expandir este alcance hacia la resolución de problemas complejos.

En particular, los graders integrados a entornos como Jupyter Notebook y Google Colab han permitido evaluar tanto la exactitud numérica como la lógica del código. Herramientas como \texttt{nbgrader} han demostrado efectividad en cursos de programación y análisis de datos \parencite{blank2017}, destacando la importancia de retroalimentación detallada y oportuna. Sin embargo, tal como señalan investigaciones recientes, aún se requiere fortalecer la capacidad de estas herramientas para evaluar procesos de razonamiento estadístico y la justificación de resultados, aspectos que trascienden la simple comparación de valores \parencite{alhaddad2024}.

En este sentido, la propuesta de integrar un grader personalizado que almacene y audite el código enviado por los estudiantes representa un aporte novedoso. Al hacerlo, se logra no solo verificar respuestas correctas o incorrectas, sino también analizar el proceso de resolución seguido por el estudiante, aumentando la transparencia y equidad en la evaluación.

\subsection{Arquitectura Tecnológica para la Educación }
El diseño de entornos educativos digitales exige arquitecturas escalables, flexibles y seguras. En este marco, los microservicios se posicionan como la opción preferida al permitir separar la lógica de negocio, la visualización de resultados y la gestión de datos en componentes independientes, mejorando así la mantenibilidad y la escalabilidad de los sistemas \parencite{mostefai2025}.

El uso de tecnologías como Node.js y Express.js en el backend y React.js en el frontend posibilita una experiencia ágil e interactiva \parencite{brown2019, banks2017}. De manera complementaria, herramientas de contenedorización como Docker facilitan la portabilidad y reproducibilidad de los entornos \parencite{merkel2014}, mientras que las bases de datos relacionales como MariaDB garantizan integridad y trazabilidad de la información académica.

Además, durante las fases de prueba y despliegue, soluciones como Playit.gg han demostrado ser útiles para exponer servicios locales de forma segura sin necesidad de configuraciones complejas, lo que las convierte en una alternativa práctica para proyectos académicos en desarrollo \parencite{gu2022}. Estas tendencias muestran que la educación digital avanza hacia modelos más robustos y modulares que favorecen tanto la escalabilidad como la personalización de la experiencia de aprendizaje.

\section{Conclusiones}
La revisión de literatura evidencia que, aunque existen avances relevantes en plataformas como Moodle y en el desarrollo de sistemas de autoevaluación, todavía persisten vacíos en la integración de estos recursos con arquitecturas modernas que soporten la enseñanza práctica de la estadística. Los estudios recientes subrayan la necesidad de combinar LMS, herramientas de autoevaluación inteligentes y arquitecturas escalables basadas en microservicios para ofrecer experiencias de aprendizaje más completas, transparentes y equitativas.

En consecuencia, la propuesta de un entorno virtual de aprendizaje para Estadística II en la UIS no solo responde a estas limitaciones, sino que también se alinea con las tendencias actuales en educación digital, ofreciendo una solución replicable y escalable donde la retroalimentación automática y la integración práctica con herramientas de análisis de datos son los ejes fundamentales.


\newpage


% ------------------------------------------------------------------------
% Bibliografía
% ------------------------------------------------------------------------


%\addcontentsline{toc}{chapter}{Referencias Bibliográficas}\newpage
%\bibliographystyle{apalike}
%\bibliography{biblio}
\addcontentsline{toc}{chapter}{Referencias Bibliográficas}

\printbibliography



\nocite{poniszewska-maranda, burns-kubernetes, torres-bosch-microservicios, armstrong2015,kubevirtio, docker2023, kubelet-doc, namespace-article}
% ------------------------------------------------------------------------
% Anexos


% ------------------------------------------------------------------------

% ------------------------------------------------------------------------
\end{document}                                          % Fin de documento
% ------------------------------------------------------------------------ 
